\documentclass{article}\usepackage{graphicx, color}
%% maxwidth is the original width if it is less than linewidth
%% otherwise use linewidth (to make sure the graphics do not exceed the margin)
\makeatletter
\def\maxwidth{ %
  \ifdim\Gin@nat@width>\linewidth
    \linewidth
  \else
    \Gin@nat@width
  \fi
}
\makeatother

\IfFileExists{upquote.sty}{\usepackage{upquote}}{}
\definecolor{fgcolor}{rgb}{0.2, 0.2, 0.2}
\newcommand{\hlnumber}[1]{\textcolor[rgb]{0,0,0}{#1}}%
\newcommand{\hlfunctioncall}[1]{\textcolor[rgb]{0.501960784313725,0,0.329411764705882}{\textbf{#1}}}%
\newcommand{\hlstring}[1]{\textcolor[rgb]{0.6,0.6,1}{#1}}%
\newcommand{\hlkeyword}[1]{\textcolor[rgb]{0,0,0}{\textbf{#1}}}%
\newcommand{\hlargument}[1]{\textcolor[rgb]{0.690196078431373,0.250980392156863,0.0196078431372549}{#1}}%
\newcommand{\hlcomment}[1]{\textcolor[rgb]{0.180392156862745,0.6,0.341176470588235}{#1}}%
\newcommand{\hlroxygencomment}[1]{\textcolor[rgb]{0.43921568627451,0.47843137254902,0.701960784313725}{#1}}%
\newcommand{\hlformalargs}[1]{\textcolor[rgb]{0.690196078431373,0.250980392156863,0.0196078431372549}{#1}}%
\newcommand{\hleqformalargs}[1]{\textcolor[rgb]{0.690196078431373,0.250980392156863,0.0196078431372549}{#1}}%
\newcommand{\hlassignement}[1]{\textcolor[rgb]{0,0,0}{\textbf{#1}}}%
\newcommand{\hlpackage}[1]{\textcolor[rgb]{0.588235294117647,0.709803921568627,0.145098039215686}{#1}}%
\newcommand{\hlslot}[1]{\textit{#1}}%
\newcommand{\hlsymbol}[1]{\textcolor[rgb]{0,0,0}{#1}}%
\newcommand{\hlprompt}[1]{\textcolor[rgb]{0.2,0.2,0.2}{#1}}%

\usepackage{framed}
\makeatletter
\newenvironment{kframe}{%
 \def\at@end@of@kframe{}%
 \ifinner\ifhmode%
  \def\at@end@of@kframe{\end{minipage}}%
  \begin{minipage}{\columnwidth}%
 \fi\fi%
 \def\FrameCommand##1{\hskip\@totalleftmargin \hskip-\fboxsep
 \colorbox{shadecolor}{##1}\hskip-\fboxsep
     % There is no \\@totalrightmargin, so:
     \hskip-\linewidth \hskip-\@totalleftmargin \hskip\columnwidth}%
 \MakeFramed {\advance\hsize-\width
   \@totalleftmargin\z@ \linewidth\hsize
   \@setminipage}}%
 {\par\unskip\endMakeFramed%
 \at@end@of@kframe}
\makeatother

\definecolor{shadecolor}{rgb}{.97, .97, .97}
\definecolor{messagecolor}{rgb}{0, 0, 0}
\definecolor{warningcolor}{rgb}{1, 0, 1}
\definecolor{errorcolor}{rgb}{1, 0, 0}
\newenvironment{knitrout}{}{} % an empty environment to be redefined in TeX

\usepackage{alltt}

\begin{document}

\section{Introduction}
\subsection{Synopsis}
Package {\tt cgmisc} contains miscellaneous functions, hopefully useful for extending genome-wide association study (GWAS) analyses. 

\subsection{Getting help}
Like every other R function, the functions provided in this package are documented in the standard R-help (Rd) format and can be easily accessed by issuing {\tt help()} or its shorter version, {\tt ?} function. For instance, if you want to get more information on how to use the {\tt clump.markers()} function, type either {\tt help(clumpmarkers)} or {\tt ?clump.markers} and press return/enter. To see this document from within R you type {\tt vignette('cgmisc')}. 

\subsection{Purpose of this document}
This document aims at presenting how to use functions provided in this package in a typical GWAS data analyses workflow. It is, however, not pretending to be a GWAS tutorial as such.

\subsection{Conventions}
\begin{itemize}
  \item{All R commands are written in terminal type: {\tt myfun(foo=T, bar=54)}}
  \item{In the above example: {\tt myfun} is a \textit{function} and both {\tt foo} and {\tt bar} are its \textit{arguments}}
\end{itemize}

\section{Working with {\tt cgmisc}}
\subsection{Installation}
In order to install {\tt cgmisc}, you either use one of the R GUIs (native R GUI, RStudio etc.) or type the following command:
\begin{knitrout}
\definecolor{shadecolor}{rgb}{0.969, 0.969, 0.969}\color{fgcolor}\begin{kframe}
\begin{alltt}
\hlfunctioncall{install.packages}(\hlstring{"cgmisc"}, repos = \hlstring{""})
\end{alltt}
\end{kframe}
\end{knitrout}

\noindent Functions in the {\tt cgmisc} package often complement or use {\tt GenABEL} \ref{aulchenko} package functions and data structures. {\tt GenABEL} is an excellent and widely-used R package for performing genome-wide association studies and much more... Therefore {\tt GenABEL} will be loaded automagically. If for some mysterious reason this does not happen, you can install and load {\tt GenABEL} by typing:
\begin{knitrout}
\definecolor{shadecolor}{rgb}{0.969, 0.969, 0.969}\color{fgcolor}\begin{kframe}
\begin{alltt}
\hlfunctioncall{install.packages}(\hlstring{"GenABEL"})
\hlfunctioncall{require}(\hlstring{"GenABEL"})
\end{alltt}
\end{kframe}
\end{knitrout}

\noindent You load {\tt cgmisc} package in exactly the same way (both {\tt require} and {\tt library} will do):
\begin{knitrout}
\definecolor{shadecolor}{rgb}{0.969, 0.969, 0.969}\color{fgcolor}\begin{kframe}
\begin{alltt}
\hlfunctioncall{require}(\hlstring{"cgmisc"})
\hlfunctioncall{library}(\hlstring{"cgmisc"})  # Alternative to require
\end{alltt}
\end{kframe}
\end{knitrout}

\end{document}
