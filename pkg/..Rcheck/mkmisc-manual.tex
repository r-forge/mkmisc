\nonstopmode{}
\documentclass[a4paper]{book}
\usepackage[times,inconsolata,hyper]{Rd}
\usepackage{makeidx}
\usepackage[utf8,latin1]{inputenc}
% \usepackage{graphicx} % @USE GRAPHICX@
\makeindex{}
\begin{document}
\chapter*{}
\begin{center}
{\textbf{\huge Package `mkmisc'}}
\par\bigskip{\large \today}
\end{center}
\begin{description}
\raggedright{}
\item[Type]\AsIs{Package}
\item[Title]\AsIs{What the package does (short line)}
\item[Version]\AsIs{1.0}
\item[Date]\AsIs{2013-03-20}
\item[Author]\AsIs{Who wrote it}
\item[Maintainer]\AsIs{Who to complain to }\email{yourfault@somewhere.net}\AsIs{}
\item[Description]\AsIs{More about what it does (maybe more than one line)}
\item[License]\AsIs{What license is it under?}
\end{description}
\Rdcontents{\R{} topics documented:}
\inputencoding{utf8}
\HeaderA{mkmisc-package}{A package containing miscellaneous data analyses functions developed by Marcin Kierczak.}{mkmisc.Rdash.package}
\aliasA{mkmisc}{mkmisc-package}{mkmisc}
\keyword{package}{mkmisc-package}
%
\begin{Description}\relax
This package is a collection of various data analysis and visualization functions I have developed in my work. Primarily these are functions used in computational genetics.
\end{Description}
%
\begin{Details}\relax

\Tabular{ll}{
Package: & mkmisc\\{}
Type: & Package\\{}
Version: & 1.0\\{}
Date: & 2013-03-20\\{}
License: & GPL (>= 2)\\{}
}
Currently the package contains the following functions:\\{}
plotMnhattanLD - plot linkage disequilibrium on your Manhattan plot
\end{Details}
%
\begin{Author}\relax
Marcin Kierczak

Maintainer: Marcin Kierczak <marcin.kierczak@slu.se>
\end{Author}
\inputencoding{utf8}
\HeaderA{plot.manhattan.LD}{A function to plot LD and MAF on Manhattan plot}{plot.manhattan.LD}
\keyword{Manhattan plot}{plot.manhattan.LD}
\keyword{LD}{plot.manhattan.LD}
\keyword{MAF}{plot.manhattan.LD}
\keyword{GWAS}{plot.manhattan.LD}
%
\begin{Description}\relax
This is a function for ploting color-coded linkage disequilibrium on a Manhattan plot. LD is relative to a selected marker. In addition, minor allele frequency (MAF) will be plotted below Manhattan. 
\end{Description}
%
\begin{Usage}
\begin{verbatim}
plot.manhattan.LD(data, gwas.result, chr, region, index.snp, p.value = 0.05, bonferroni = T, mafThreshold = 0.05)
\end{verbatim}
\end{Usage}
%
\begin{Arguments}
\begin{ldescription}
\item[\code{data}] 
an object of the gwaa.data class (GenABEL internal class)

\item[\code{gwas.result}] 
an object of the gwaa.scan class containing association testing results (GenABEL class)

\item[\code{chr}] 
the chromosome to be plotted

\item[\code{region}] 
a vector of genomic coordinates to be plotted (chromosome-wise coords)

\item[\code{index.snp}] 
a reference marker. LD will be computed with respect to this marker.

\item[\code{p.value}] 
a p-value threshold for plotting a threshold line

\item[\code{bonferroni}] 
a logical indicating whether Bonferroni correction shall be used

\item[\code{mafThreshold}] 
a threshold for plotting MAF threshold line

\end{ldescription}
\end{Arguments}
%
\begin{Value}
no return value, a plot will be displayed
\end{Value}
%
\begin{Author}\relax
Marcin Kierczak
\end{Author}
%
\begin{Examples}
\begin{ExampleCode}
##---- Should be DIRECTLY executable !! ----
##-- ==>  Define data, use random,
##--	or do  help(data=index)  for the standard data sets.

## The function is currently defined as
function (data, gwas.result, chr, region, index.snp, p.value = 0.05, 
    bonferroni = T, mafThreshold = 0.05) 
{
    shift <- 2.3
    topMargin <- 0
    getMAF <- function(data, region) {
        summ <- summary(gtdata(data)[, region])
        tmp_maf <- (summ$P.22 + 0.5 * summ$P.12)/summ$NoMeasured
        tmp_ma_count <- summ$P.22 + 0.5 * summ$P.12
        tmp_het <- summ$P.12/summ$NoMeasured
        maf <- data.frame(snp = as.character(row.names(summ)), 
            chr = summ$Chromosome, pos = map(gtdata(data)[, region]), 
            allele = summ$A2, maf = tmp_maf, maCnt = tmp_ma_count, 
            heterozygosity = tmp_het)
    }
    startCoord <- region[1]
    stopCoord <- region[2]
    myChromosome <- data@gtdata[, which(data@gtdata@chromosome == 
        chr)]
    region <- which(myChromosome@map >= startCoord & myChromosome@map <= 
        stopCoord)
    r2matrix <- r2fast(myChromosome, snpsubset = region)
    r2matrix[lower.tri(r2matrix)] <- t(r2matrix)[lower.tri(r2matrix)]
    markers <- which(data@gtdata@snpnames %in% names(region))
    markers.coords <- data@gtdata@map[markers]
    idx.marker <- which(data@gtdata@snpnames == index.snp)
    idx.marker.coords <- data@gtdata@map[idx.marker]
    pvals <- -log10(gwas.result@results$P1df[markers])
    plot(markers.coords, pvals, type = "n", xlab = "Position (Mb)", 
        ylab = expression(-log[10](p - value)), ylim = c(-shift, 
            max(pvals) + 3), axes = F)
    abline(h = seq(0.5, max(pvals) + topMargin, 0.5), col = "grey", 
        lty = 3)
    r2vec <- r2matrix[index.snp, ]
    r2vec[is.na(r2vec)] <- -1
    r2col <- cut(r2vec, breaks = c(1, 0.8, 0.6, 0.4, 0.2, 0, 
        -1), labels = rev(c("#9E0508", "tomato", "chartreuse3", 
        "cyan3", "navy", "black")), include.lowest = T)
    r2pch <- rep(19, length(r2col))
    r2pch[which(r2col == "black")] <- 1
    points(markers.coords, -log10(gwas.result@results$P1df[markers]), 
        col = as.character(r2col), pch = r2pch, cex = 0.8)
    if (bonferroni) {
        p.value <- -log10(p.value/nids(data))
        abline(h = -log10(p.value), col = "red", lty = 2)
    }
    legend(startCoord + 10, max(pvals) + 1, legend = c("0.8 - 1.0", 
        "0.6 - ", "0.4 -", "0.2 -", "0.0 -"), pch = 19, bty = "n", 
        col = c("#9E0508", "tomato", "chartreuse3", "cyan3", 
            "navy"), cex = 0.7, title = expression(r^2))
    maf <- getMAF(data, region)
    lines(markers.coords, 4 * maf$maf - shift, col = "#2C7FB8")
    abline(h = mafThreshold * 4 - shift, col = rgb(1, 0, 0, 1), 
        lty = 2)
    step <- (stopCoord - startCoord)/5
    axis(1, at = seq(startCoord, stopCoord, by = step), labels = format(seq(startCoord, 
        stopCoord, by = step)/1e+06, scientific = F, digits = 3))
    axis(2, at = 0:(max(pvals) + topMargin + 1))
    axis(4, at = c(2 - shift, 1 - shift, 0.2 - shift, 0 - shift), 
        labels = c(0.5, 0.25, 0.05, 0))
    mtext("MAF", side = 2, at = 0.2 - shift, outer = F)
  }
\end{ExampleCode}
\end{Examples}
\printindex{}
\end{document}
